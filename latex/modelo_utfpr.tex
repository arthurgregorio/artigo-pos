%% Exemplo de utilizacao do estilo de formatacao normas-utf-tex (http://normas-utf-tex.sourceforge.net)
%% Autores: Hugo Vieira Neto (hvieir@utfpr.edu.br)
%%          Diogo Rosa Kuiaski (diogo.kuiaski@gmail.com)
%% Colaboradores:
%%          C�zar M. Vargas Benitez <cesarvargasb@gmail.com>
%%          Marcos Talau <talau@users.sourceforge.net>


\documentclass[openright]{normas-utf-tex} %openright = o capitulo comeca sempre em paginas impares
%\documentclass[oneside]{normas-utf-tex} %oneside = para dissertacoes com numero de paginas menor que 100 (apenas frente da folha) 


\usepackage[alf,abnt-emphasize=bf,bibjustif,recuo=0cm, abnt-etal-cite=2, abnt-etal-list=99]{abntcite} %configuracao correta das referencias bibliograficas.

\usepackage[brazil]{babel} % pacote portugues brasileiro
\usepackage[latin1]{inputenc} % pacote para acentuacao direta
\usepackage{amsmath,amsfonts,amssymb} % pacote matematico
\usepackage{graphicx} % pacote grafico
\usepackage{times} % fonte times

%Podem utilizar GEOMETRY{...} para realizar pequenos ajustes das margens. Onde, left=esquerda, right=direita, top=superior, bottom=inferior. P.ex.:
%\geometry{left=3.0cm,right=1.5cm,top=4cm,bottom=1cm} 

% ---------- Preambulo ----------
\instituicao{UNIVERSIDADE TECNOL�GICA FEDERAL DO PARAN�
C�MPUS CORN�LIO PROC�PIO} % nome da instituicao
\programa{DIRETORIA DE GRADUA��O E EDUCA��O PROFISSIONAL
ESPECIALIZA��O EM TECNOLOGIA JAVA} % nome do programa
%\area{Inform\'atica} % [Engenharia Biom\'edica] ou [Inform\'atica Industrial] ou [Telem\'atica]

\documento{\MakeUppercase{Trabalho de conclus�oo de curso}} % [Disserta\c{c}\~ao] ou [Tese]
\nivel{Especializa��o} % [Mestrado] ou [Doutorado]
\titulacao{Especialista} % [Mestre] ou [Doutor]

\titulo{\MakeUppercase{Proposta de arquitetura web para aplica��es Java}} % titulo do trabalho em portugues
\title{\MakeUppercase{Web architecture proposal for Java applications}} % titulo do trabalho em ingles

\autor{Arthur Pereira Greg�rio} % autor do trabalho
\cita{Gregorio, Arthur} % sobrenome (maiusculas), nome do autor do trabalho

\palavraschave{arquitetura, java, jee, frameworks, proposta} % palavras-chave do trabalho
\keywords{architecture, java, jee, frameworks, proposal} % palavras-chave do trabalho em ingles

\comentario{Trabalho de Conclus�o de Curso de gradua��o, apresentado � disciplina ???, do curso de Especializa��o em tecnologia Java da Universidade Tecnol�gica Federal do Paran�, como requisito parcial para a obten��o do t�tulo de Especialista.}

\orientador{Andr� Roberto Ortoncelli} % nome do orientador do trabalho
%\orientador[Orientadora:]{Nome da Orientadora} % <- no caso de orientadora, usar esta sintaxe
%\coorientador{Nome do Co-orientador} % nome do co-orientador do trabalho, caso exista
%\coorientador[Co-orientadora:]{Nome da Co-orientadora} % <- no caso de co-orientadora, usar esta sintaxe
%\coorientador[Co-orientadores:]{Nome do Co-orientador} % no caso de 2 co-orientadores, usar esta sintaxe
%\coorientadorb{Nome do Co-orientador 2}	% este comando inclui o nome do 2o co-orientador

\local{\MakeUppercase{Foz do Igua�u}} % cidade
\data{\the\year} % ano automatico

%---------- Inicio do Documento ----------
\begin{document}

\capa % geracao automatica da capa
\folhaderosto % geracao automatica da folha de rosto
%\termodeaprovacao % <- ainda a ser implementado corretamente

% dedicatoria (opcional)
\begin{dedicatoria}
Dedico este trabalho a minha familia, em especial minha esposa, seu apoio e paci�ncia em meus momentos de estudo foi essencial para a conclus�o deste trabalho.
\end{dedicatoria}

%resumo
\begin{resumo}
~resumo~
\end{resumo}

%abstract
\begin{abstract}
~Abstract~
\end{abstract}

% listas (opcionais, mas recomenda-se a partir de 5 elementos)
\listadefiguras % geracao automatica da lista de figuras
%\listadetabelas % geracao automatica da lista de tabelas
\listadesiglas % geracao automatica da lista de siglas
%\listadesimbolos % geracao automatica da lista de simbolos

% sumario
\sumario % geracao automatica do sumario

%---------- Inicio do Texto ----------
% recomenda-se a escrita de cada capitulo em um arquivo texto separado (exemplo: intro.tex, fund.tex, exper.tex, concl.tex, etc.) e a posterior inclusao dos mesmos no mestre do documento utilizando o comando \input{}, da seguinte forma:
%\input{intro.tex}
%\input{fund.tex}
%\input{exper.tex}
%\input{concl.tex}

%---------- Introdu��o 
\include{./Capitulos/capitulo1} 
%---------- Fundamenta��o 
\include{./Capitulos/capitulo2} 

\include{./Capitulos/capitulo3}
%---------- Resultados
\include{./Capitulos/capitulo4} 
%---------- Conclus�o 
\include{./Capitulos/capitulo5} 

%---------- Referencias ----------
\bibliography{./Referencias/reflatex} 

%---------- Apendices (opcionais) ----------
%\apendice
%\chapter{Nome do Ap\^endice}

%Use o comando {\ttfamily \textbackslash apendice} e depois comandos {\ttfamily %\textbackslash chapter\{\}} para gerar t\'itulos de ap\^en-dices.

% ---------- Anexos (opcionais) ----------
%\anexo
%\chapter{Nome do Anexo}
%Use o comando {\ttfamily \textbackslash anexo} e depois comandos {\ttfamily %\textbackslash chapter\{\}} para gerar t\'itulos de anexos.

% --------- Lista de siglas --------
%\textbf{* Observa\c{c}\~oes:} a lista de siglas nao realiza a ordenacao das siglas em ordem alfabetica
% Em breve isso sera implementado, enquanto isso:
%\textbf{Sugest\~ao:} crie outro arquivo .tex para siglas e utilize o comando \sigla{sigla}{descri\c{c}\~ao}.
%Para incluir este arquivo no final do arquivo, utilize o comando \input{arquivo.tex}.
%Assim, Todas as siglas serao geradas na ultima pagina. Entao, devera excluir a ultima pagina da versao final do arquivo
% PDF do seu documento.

%-------- Citacoes ---------
% - Utilize o comando \citeonline{...} para citacoes com o seguinte formato: Autor et al. (2011).
% Este tipo de formato eh utilizado no comeco do paragrafo. P.ex.: \citeonline{autor2011}

% - Utilize o comando \cite{...} para citacoeses no meio ou final do paragrafo. P.ex.: \cite{autor2011}

%-------- Titulos com nomes cientificos (titulo, capitulos e secoes) ----------
% Regra para escrita de nomes cientificos:
% Os nomes devem ser escritos em italico, 
%a primeira letra do primeiro nome deve ser em maiusculo e o restante em minusculo (inclusive a primeira letra do segundo nome).
% VEJA os exemplos abaixo.
% 
% 1) voce nao quer que a secao fique com uppercase (caixa alta) automaticamente:
%\section[nouppercase]{\MakeUppercase{Estudo dos efeitos da radiacao ultravioleta C e TFD em celulas de} {\textit{Saccharomyces boulardii}}
%
% 2) por padrao os cases (maiusculas/minuscula) sao ajustados automaticamente, voce nao precisa usar makeuppercase e afins.
% \section{Introducao} % a introducao sera posta no texto como INTRODUCAO, automaticamente, como a norma indica.

\end{document}