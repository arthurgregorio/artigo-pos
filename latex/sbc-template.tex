\documentclass[12pt]{article}

\usepackage{sbc-template}

\usepackage{graphicx,url}

%\usepackage[brazil]{babel}   
\usepackage[latin1]{inputenc}  

     
\sloppy

\title{Proposta de arquitetura JEE para aplica��es Java web}

\author{Arthur Pereira Greg�rio}


\address{Universidade Tecnol�gica federal do Paran�
  (UTFPR)\\
  ~endereco aqui ?!~
\nextinstitute
  Departamento de Sistemas e Computa��o\\
  Universidade Regional de Blumenal (FURB) -- Blumenau, SC -- Brazil
  \email{\{nedel,flavio\}@inf.ufrgs.br, R.Bordini@durham.ac.uk,
  jomi@inf.furb.br}
}

\begin{document} 

\maketitle

\begin{abstract}

\end{abstract}
     
\begin{resumo} 
  
\end{resumo}


\section{Introdu��o}

De acordo com \cite{xavier2013} arquitetura \'e a defini��o dos elementos que comp\~oem uma estrutura e como estes v\~ao se relacionar, assim podemos dizer que a cria\c{c}\~o de uma arquitetura de software \'e algo que exige do arquiteto uma s\'erie de conhecimentos espec\'ificos sobre como estes componentes funcionam ou como podem interagir entre s\'i.

Definir uma arquitetura n�o \'e algo simples, pois mais importante do que saber integrar os compomentes selecionados, \'e saber identificar quais as poss\'iveis escolhas e quais os fatores que influenciam nestas escolhas \cite{xavier2013}. 

\section{Trabalhos relacionados}


\section{Desenvolvimento}


\section{Conclus�o}


\section{Refer�ncias}

\bibliographystyle{sbc}
\bibliography{sbc-template}

\end{document}
